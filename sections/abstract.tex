\begin{abstract}
	While gradient descent is ubiquitous in Machine Learning, there is no
	adaptive way to select a learning rate yet. This forces practitioners to do
	"hyperparameter tuning". We review how optimization schemes can be motivated
	using Taylor approximations and develop intuition why this results in unknown
	hyperparameters. We then replace the Taylor approximation with a statistical
	Best Linear Unbiased Estimator (BLUE) and derive gradient descent again. But
	this time with calculable learning rates.
\end{abstract}