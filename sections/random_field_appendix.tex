\section{Random Fields}



\begin{definition}[Variogram]
	\label{def: variogram}
	A random field \(Z\), for which
	\[
		\tilde{Z}(x) = Z(x+\step)- Z(x)
	\]
	is stationary for every \(\step\), is called \emph{intrinsically stationary}.
	For any intrinsically stationary random field, the \emph{(centred) variogram} is
	defined as
	\[
		\variogram(\step)	= \tfrac12\Var(Z(x+\step)-Z(x)),
	\]
	which is independent of \(x\) due to stationarity of \(\tilde{Z}\). For the
	uncentered variogram use uncentered second moments instead of the variance.
\end{definition}

\begin{lemma}[Covariance from Variogram]
	\label{lem: covariance from variogram}
	If \(Z\) is intrinsically stationary we have for \(Y_s := Z - Z(s)\)
	and
	\[
		\C_s(x,y) := \Cov(Y_s(x), Y_s(y))
	\]
	an explicit representation using the variogram
	\begin{align*}
		\C_s(x,y)
		&= \variogram(x-s) + \variogram(y-s) - \variogram(x-y)\\
		\overset{\text{symm.}}&= \variogram(x-s) + \variogram(y-s) - \variogram(y-x)
	\end{align*}
\end{lemma}
\begin{remark}[Antisymmetry]\label{rem: grad variogram antisymmetric}
	We therefore have \(\variogram(x-y) = \variogram(y-x)\) or in other words
	the variogram is symmetric. So if it is differentiable, then
	\(\grad\variogram\) is antisymmetric with \(\grad\variogram(0)=0\).
\end{remark}
\begin{proof}[Proof (by Schlather?)]\fxnote{source?}
	Assume \(Z\) is centred without loss of generality, otherwise center \(Z\)
	resulting in \(\tilde{Z}\) and notice that \(\tilde{Y}_s\) is the centered
	version of \(Y_s\)
	\begin{align*}
		\C_s(x,y) 
		&= \E[(Z(x) - Z(s))(Z(y)-Z(s))]\\
		&= \E[(Z(x) - Z(y) + Z(y) - Z(s))(Z(y) - Z(x) + Z(x) - Z(s))]\\
		&=\begin{aligned}[t]
			-&\E[(Z(x) - Z(y))^2] &&+ \underbrace{\E[(Z(x)-Z(y))(Z(x)-Z(s))]}_{=:\Rom{1}}\\
			+&\C_s(y,x)	&&+ \underbrace{\E[(Z(y) - Z(s))(Z(y) - Z(x))]}_{=:\Rom{2}}
		\end{aligned}
	\end{align*}
	And we have
	\begin{align*}
		\Rom{1}
		&=\E[(Z(x)- Z(s) + Z(s) - Z(y))(Z(x) - Z(s))]\\
		&= 2\variogram(x-s) - \C_s(y,x)
	\end{align*}
	and similarly \(\Rom{2} = 2\variogram(y-s) - \C_s(y,x)\). Together this
	results in
	\[
		\C_s(x,y) = -2\variogram(x-y) + 2\variogram(x-s) - \C_s(y,x) + 2\variogram(y-s) - \cancel{\C_s(y,x)} + \cancel{\C_s(y,x)}.
	\]
	Reordering and symmetry of the covariance results in the claim.
\end{proof}

\subsection{Derivatives of Random Fields}

\begin{lemma}[Covariance and Derivative]
	\label{lem: covariance of derivative}
	We have
	\begin{align*}
		\Cov(\partial_i\Loss(x), \Loss(y)) &= \partial_{x_i} \C(x,y)\\
		\Cov(\partial_i\Loss(x), \partial_j \Loss(y)) &= \partial_{x_i y_j} \C(x,y)
	\end{align*}
\end{lemma}
\begin{remark}\label{rem: covariance uncorrelated}
	For stationary random fields \(\C(x,y)=\C(x-y)\), the covariance
	function \(\C\) is symmetric, i.e. \(\C(h)=\C(-h)\)
	and therefore the derivative antisymmetric, i.e.
	\(\partial_i\C(-h)=-\partial_i\C(h)\). In particular
	\begin{equation*}
		\Cov(\grad\Loss(x), \Loss(x)) = \grad \C(0)=0.
	\end{equation*}
\end{remark}
\begin{proof}
	
\end{proof}
\begin{corollary}[Gaussian Case]
	If \(\Loss\) is a stationary gaussian random field, \(\Loss(x)\) and
	\(\grad\Loss(x)\) are independent multivariate gaussian for every \(x\).
\end{corollary}

\begin{lemma}[Variogram and Derivative]\label{lem: variogram and derivative}
	Assuming \(Z\) is an intrinsic, stationary, differentiable random field,
	we have
	\begin{enumerate}
		\item \(
			\Cov(Z(x+\step)-Z(x), \grad Z(x)) = \grad\variogram(\step)
		\)
		\item \(\Cov(\grad Z(x), \grad Z(y)) = \grad^2\variogram(x-y)\)
	\end{enumerate}
\end{lemma}
\begin{proof}
	\begin{enumerate}[wide]
	\item We have
	\begin{equation*}
		\partial_i Z(x) = \lim_{h\to 0}\frac{Z(x + he_i) - Z(s) -(Z(x)-Z(s))}{h}
		= \partial_i Y_s(x) \quad \forall x,s
	\end{equation*}
	and therefore
	\begin{equation*}
		\Cov(\partial_i Z(x), Z(x+\step)-Z(x))
		\overset{\text{Lem~\ref{lem: covariance of derivative}}}=
		\partial_{x_i}\C_x(x, x+\step)
		= \underbrace{\partial_i\variogram(0)}_{
			=0 \mathrlap{\text{ (Remark~\ref{rem: grad variogram antisymmetric})}}
		}
		- \partial_i\variogram(-\step),
	\end{equation*}
	because the variogram is symmetric and the derivative therefore antisymmetric.

	\item Here we use \(\partial_i Z(x) = \partial_i Y_s(x)\) again
	\begin{align*}
		\Cov(\partial_i Z(x), \partial_j Z(y))
		&= \Cov(\partial_i Y_s(x), \partial_j Y_s(y))\\
		\overset{\text{Lem~\ref{lem: covariance of derivative}}}&=
		\partial_{y_j x_i} \C_s(x,y)\\
		\overset{\text{Lem~\ref{lem: covariance from variogram}}}&=
		\partial_{y_j x_i} [\variogram(x-s) + \variogram(y-s) - \variogram(x-y)]\\
		&= \partial_{y_j} [\partial_i\variogram(x-s) - \partial_i\variogram(x-y)]\\
		&= \partial_{ij}\variogram(x-y).
		\qedhere
	\end{align*}

	\end{enumerate}

\end{proof}
