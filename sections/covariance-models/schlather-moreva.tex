
\subsection{Schlather-Moreva}

To bridge bounded variograms (stationary random fields) and
unbounded variograms (intrinsically stationary but not stationary random fields)
\textcite{schlatherParametricModelBridging2017} suggest
\begin{equation*}
	\variogram_{\alpha,\beta}(\step)
	= \frac{(1+\|\step\|^\alpha)^{\beta/\alpha}-1}{2^{\beta/\alpha}-1}
	\quad \alpha\in (0,2],\ \beta\in(-\infty, 2].
\end{equation*}
Since the model only results in a differentiable random field for \(\alpha=2\),
we will only consider \(\variogram_\beta(\step) := \variogram_{2,\beta}(\step) = \phi(\|\step\|^2)\),
with
\begin{equation*}
	\phi(x) = \frac{(1+x)^{\beta/2} -1}{2^{\beta/2}-1}.
\end{equation*}
Taking its derivative results in
\begin{equation*}
	\phi'(x) = \frac{\tfrac\beta2 (1+x)^{\tfrac\beta2 -1}}{2^{\beta/2}-1}
\end{equation*}
which implies that
\begin{align*}
	\hat{\lr} = \argmax_\lr \lr\frac{\phi'(\lr^2)}{\phi'(0)}
	= \argmax_\lr \lr(1+\lr^2)^{\tfrac\beta2 -1}.
\end{align*}
For \(\beta \ge 1\) this function is monotonically increasing in \(\lr\) and
we would therefore have \(\hat{\lr} = \infty\), so only the case \(\beta <1\)
is interesting\fxwarning{explanation for \(\beta\ge 1\)?}. Taking the derivative with regard to
\(\lr\) results in
\begin{align*}
	0 \overset!=\frac{d}{d\lr}
	&= (1+\lr^2)^{\frac\beta2 -2} \left[(1+\lr^2) - 2\lr^2\Big(1-\frac\beta2\Big)\right]\\
	&= (1+\lr^2)^{\frac\beta2 -2} \left[1 - \lr^2(1-\beta)\right]
\end{align*}
So we get
\[
	\hat{\lr} = \frac1{\sqrt{1-\beta}}.
\]
As we can see that the derivative is larger than zero before and smaller
afterwards it is indeed a maximum.
