
\subsection{Schlather-Moreva}

To bridge bounded variograms (stationary random fields) and
unbounded variograms (intrinsically stationary but not stationary random fields)
\textcite{schlatherParametricModelBridging2017} suggest
\begin{equation*}
	\variogram_{\alpha,\beta}(\step)
	= \frac{(1+\|\step\|^\alpha)^{\beta/\alpha}-1}{2^{\beta/\alpha}-1}
	\quad \alpha\in (0,2],\ \beta\in(-\infty, 2].
\end{equation*}
Since the model only results in a differentiable random field for \(\alpha=2\),
we will only consider \(\variogram_\beta(\step) := \variogram_{2,\beta}(\step) = \phi(\|\step\|^2)\),
with
\begin{equation*}
	\phi(x) = \frac{(1+x)^{\beta/2} -1}{2^{\beta/2}-1}.
\end{equation*}
Taking its derivative results in
\begin{equation*}
	\phi'(x) = \frac{\tfrac\beta2 (1+x)^{\tfrac\beta2 -1}}{2^{\beta/2}-1}
\end{equation*}
which implies with Theorem~\ref{thm: bayesian descent} that
\begin{align*}
	\hat{\lr} = \argmax_\lr \lr\frac{\phi'(\lr^2)}{\phi'(0)}
	= \argmax_\lr \lr(1+\lr^2)^{\tfrac\beta2 -1}.
\end{align*}
For \(\beta \ge 1\) this function is monotonically increasing in \(\lr\) and
we would therefore have \(\hat{\lr} = \infty\), so only the case \(\beta <1\)
is interesting\fxwarning{explanation for \(\beta\ge 1\)?}. Taking the derivative with regard to
\(\lr\) results in
\begin{align*}
	0 \overset!=\frac{d}{d\lr}
	&= (1+\lr^2)^{\frac\beta2 -2} \left[(1+\lr^2) - 2\lr^2\Big(1-\frac\beta2\Big)\right]\\
	&= (1+\lr^2)^{\frac\beta2 -2} \left[1 - \lr^2(1-\beta)\right]
\end{align*}
As we can see that the derivative is larger than zero before and smaller
afterwards it is indeed a maximum.
So we get

\begin{theorem}[Intrinsically Stationary]
	Assuming \(\Loss\) is centered and has variogram \(\variogram_{2,\beta}\)
	with \(\beta\in(-\infty, 1]\), we have
	\begin{align*}
		\step(\hat{\lr})
		&= \argmin_{d}
		\BLUE[\Loss(\param + \step) - \Loss(\param)\mid \grad\Loss(\param)]\\
		\hat{\lr}
		&= \frac1{\sqrt{1-\beta}}.
	\end{align*}
\end{theorem}


For \(\beta<0\) the variogram is bounded. So we can define
\[
	\C_\beta(\lr) := c_\beta - \variogram_\beta(\lr)
\]
with \(c_\beta := (1-2^{\beta/2})^{-1} =
\lim_{\lr\to\infty}\variogram_\beta(\variogram)\), and a (stationary) random
field with this covariance function will have variogram \(\variogram_\beta\).


\begin{theorem}[Stationary]
	Assuming \(\Loss\) is centered, has covariance \(\C_\beta\) with
	\(\beta\in(-\infty, 0)\), then we have for \(\Loss(\param)\le 0\)
	\begin{align*}
		\step(\hat{\lr})
		&= \argmin_{d}
		\BLUE[\Loss(\param + \step) \mid \Loss(\param), \grad\Loss(\param)]\\
		\hat{\lr}
		&= \Root_\lr\left(
			-1 + \frac{\beta\Loss(\param)}{\|\grad\Loss(\param)\|}\lr
			+(1-\beta)\lr^2 + \frac{\beta\Loss(\param)}{\|\grad\Loss(\param)\|}\lr^3
		\right).
	\end{align*}
	The unique root can be found either by solving the polynomial of third degree
	(e.g. using Cardano's method) or by bisection as the root is in \([0,
	1/\sqrt{1-\beta}]\) and the function is monotonously increasing.
\end{theorem}

	\fxnote{
		case \(\Loss(\theta)>0\): from plotting it looks like the first
		critical point is still a maximum
	}

\begin{proof}
	By Theorem~\ref{thm: bayesian descent}	we have
	\begin{equation*}
		\hat{\lr}	
		= \argmin_{\lr}\frac{\sqC(\lr^2)}{\sqC(0)} \frac{\Loss(\param)}{\|\grad\Loss(\param)\|}
		-  \lr\frac{\sqC'(\lr^2)}{\sqC'(0)}
	\end{equation*}
	for \(\sqC(\lr^2) = \C_\beta(\lr) = c_\beta - \phi(\lr^2)\). Therefore need
	to minimize
	\begin{equation*}
		\frac{c_\beta - \phi(\lr^2)}{c_\beta - \phi(0)}
		\frac{\Loss(\param)}{\|\nabla\Loss(\param)\|}
		- \lr\frac{\phi'(\lr^2)}{\phi'(0)}
		= (1+\lr^2)^{\beta/2} 
		\frac{\Loss(\param)}{\|\nabla\Loss(\param)\|}
		- \lr(1+\lr^2)^{\frac\beta2-1}\\
	\end{equation*}
	The first order condition is hence
	\begin{align*}
		0\overset!=\frac{d}{d\lr}
		&= \tfrac\beta2(1+\lr^2)^{\frac\beta2-1}2\lr 
		\tfrac{\Loss(\param)}{\|\nabla\Loss(\param)\|}
		- (1+\lr^2)^{\frac\beta2 -2} \left[1 - \lr^2(1-\beta)\right]\\
		&= (1+\lr^2)^{\frac\beta2-2}\underbrace{\left[
			\beta\lr\tfrac{\Loss(\param)}{\|\nabla\Loss(\param)\|}(1+\lr^2)
			- [1-\lr^2(1-\beta)]
		\right]}_{
			= -1 + \frac{\beta\Loss(\param)}{\|\grad\Loss(\param)\|}\lr
			+(1-\beta)\lr^2 + \frac{\beta\Loss(\param)}{\|\grad\Loss(\param)\|}\lr^3
		}
	\end{align*}
	Since \(\Loss(\param), \beta\le0\) all coefficients of the polynomial
	are positive except for the shift. Therefore the polynomial starts out
	at \(-1\) in zero and only increases from there. Therefore there exists
	a unique positive critical point which is a minimum.
\end{proof}
