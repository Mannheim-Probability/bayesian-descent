\section{Upper Bounds}

Instead of minimizing the \(\BLUE\) directly, like one might minimize the
Taylor approximation directly, we could also add an error estimation to
get some form of upper bound. Since we are working with Random Fields, these
bounds will be stochastic. E.g. of the form
\begin{align*}
	\Pr(\Loss(\param + \step) \ge \underbrace{
		\BLUE[\Loss(\param + \step)] + \delta_\epsilon(\step)
		}_{=: \hat{\Loss}(\param + \step)})
	\le \epsilon
\end{align*}
we could then optimize over \(\hat{\Loss}(\param + \step)\). How could one
find such a \(\delta_\epsilon\)? One way is the application of the
Markov inequality
\begin{align*}
	\Pr(\Loss(\param + \step) \ge \hat{\Loss}(\param + \step))
	&=\Pr(\Loss(\param + \step) - \BLUE[\Loss(\param + \step)] \ge \delta_\epsilon(\step))\\
	&\le\Pr(|\Loss(\param + \step) - \BLUE[\Loss(\param + \step)] | \ge \delta_\epsilon(\step))\\
	&\le \frac{\E[\|\Loss(\param + \step) - \BLUE[\Loss(\param + \step)]\|^2]}{\delta_\epsilon(\step)}
	\overset{!}&{=} \epsilon.
\end{align*}
A sensible selection of \(\delta_\epsilon\) is therefore
\begin{align*}
	\delta_\epsilon(\step)
	= \sqrt{\frac{\E[\|\Loss(\param + \step) - \BLUE[\Loss(\param + \step)]\|^2]}{\epsilon}}.
\end{align*}
Since the \(\BLUE\) is a minimizer of the enumerator, we can actually state
an explicit formula.
\begin{lemma}
	For a stationary, centered loss \(\Loss\) we have
	\begin{align*}
		&\E[\|\Loss(\param + \step) - \BLUE[\Loss(\param + \step)\mid \Loss(\param), \grad\Loss(\param)]\|^2]\\
		&= \C(0)\Big[1- \left(\frac{C(\step)}{C(0)}\right)^2\Big]
		- \left\langle \grad\C(\step), [-\grad^2\C(0)]^{-1} \grad\C(\step)\right\rangle
	\end{align*}
	in the isotropic case this simplifies to
	\begin{align*}
		\sqC(0)\Big[1- \left(\frac{\sqC(\|\step\|^2)}{\sqC(0)}\right)^2\Big]
		- 2\frac{\sqC'(\|\step\|^2)}{\sqC'(0)}\|\step\|^2.
	\end{align*}
	It is in particular independent of the direction of \(\step\), so the
	gradient direction is still optimal.
\end{lemma}