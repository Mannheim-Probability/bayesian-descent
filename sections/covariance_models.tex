\section{Covariance Models}

\subsection{Squared Exponential}

In this section we are going to assume \(\C(\step)=\sqC(\|\step\|^2)\) with
\begin{align*}
	\sqC(x) = c e^{-ax}.
\end{align*}
Then the \(\BLUE\) simplifies further
\begin{align*}
	\BLUE[\Loss(\param + \step)\mid \Loss(\param),\nabla\Loss(\param)]
	&= e^{-a\|\step\|^2}\Loss(\param)
	+ e^{-a\|\step\|^2}\langle \step, \nabla\Loss(\param)\rangle.
\end{align*}
Recall that the resulting optimization problem was
\begin{align*}
	\min_{\lr} \frac{\sqC(\lr^2)}{\sqC(0)}\Loss(\param)
	- \frac{\sqC'(\lr^2)}{\sqC'(0)}\lr \|\nabla\Loss(\param)\|
	=\min_{\lr} e^{-a\lr^2}
	\left[\Loss(\param) - \lr \|\nabla\Loss(\param)\|\right]
\end{align*}
This results in the first order condition
\begin{align*}
	0 \overset{!}&{=} -2a\lr e^{-a\lr^2}
	\left[\Loss(\param) - \lr \|\nabla\Loss(\param)\|\right]
	- e^{-a\lr^2}\|\nabla\Loss(\param)\|\\
	&= e^{-a\lr^2}\left[
		2a\|\nabla\Loss(\param)\|\lr^2 - 2a\Loss(\param)\lr - \|\nabla\Loss(\param)\|
	\right].
\end{align*}
Since the exponential can never be zero, we need the bracket to be zero. This
results in the quadratic equation
\begin{align*}
	0 = \lr^2 - \frac{\Loss(\param)}{\|\nabla\Loss(\param)\|}\lr - \frac{1}{2a},
\end{align*}
and therefore
\begin{theorem}
	For a centred random field \(\Loss\) with squared exponential covariance function
	\(\C(\step) = ce^{-a\|\step\|^2}\), we have
	\begin{equation*}
		\hat{\step} := -\frac{\hat{\lr}}{\|\nabla\Loss(\param)\|}\nabla\Loss(\param)
		= \argmin_{\step} \BLUE[\Loss(\param + \step)\mid \Loss(\param),\nabla\Loss(\param)]
	\end{equation*}	
	with
	\begin{equation*}
		\hat{\lr}
		= \frac{\Loss(\param)}{2\|\nabla\Loss(\param)\|}
		+ \sqrt{
			\left(\frac{\Loss(\param)}{2\|\nabla\Loss(\param)\|}\right)^2 + \frac1{2a}
		}.
	\end{equation*}
	Similarly we have
	\begin{equation*}
		-\frac{\tilde{\lr}}{\|\nabla\Loss(\param)\|}\nabla\Loss(\param)
		= \argmin_{\step} \BLUE[\Loss(\param + \step)\mid \nabla\Loss(\param)]
	\end{equation*}	
	with
	\begin{equation*}
		\tilde{\lr} = \frac1{\sqrt{2a}}.
	\end{equation*}

\end{theorem}
\begin{remark}
	Since \(\Loss\) is centered, i.e. \(\E[\Loss(\param)]\), any optimization
	scheme will soon find \(\Loss(\param)\le 0\). In that case we have
	\begin{align*}
		\hat{\lr} =
		\Big| \left\|\begin{pmatrix}
			Q/2\\ 1/\sqrt{2a}	
		\end{pmatrix}\right\|
		- \left\|\begin{pmatrix}
			Q/2 \\ 0
		\end{pmatrix}\right\|
		\Big|	
		\le \left\|\begin{pmatrix}
			Q/2\\ 1/\sqrt{2a}
		\end{pmatrix} - \begin{pmatrix}
			Q/2 \\ 0
		\end{pmatrix}
		\right\| = \frac1{\sqrt{2a}}
	\end{align*}
	with
	\begin{align*}
		Q = \frac{\Loss(\param)}{\|\nabla\Loss(\param)\|}.
	\end{align*}
	And for \(Q\to -\infty\) we have \(\hat{\lr}\to 0\).
\end{remark}

\begin{theorem}[Improvement Guarantee]
	Assuming \(\Loss(\param)< 0\), we have
	\begin{align*}
		\BLUE[\Loss(\param + \hat{\step})\mid \Loss(\param), \nabla\Loss(\param)]
		\le \Loss(\param)\underbrace{\frac14\left[
			3 + \sqrt{1+\frac{2\|\nabla\Loss(\param)\|^2}{a\Loss(\param)^2}}
		\right]}_{> 1}
		< \Loss(\param).
	\end{align*}
	While the \(\BLUE\) is unbiased and the same can therefore be said about the
	expectation of \(\Loss(\param+\hat{\step})\), this statement is not
	particularly useful as both sides are zero.
\end{theorem}
\begin{proof}
	We have
	\begin{align*}
		&\BLUE[\Loss(\param + \hat{\step})\mid \Loss(\param), \nabla\Loss(\param)]\\
		&= e^{-a\hat{\lr}^2}[\Loss(\param) - \hat{\lr}\|\nabla\Loss(\param)\|]\\
		&=e^{-a\hat{\lr}^2}\left[
			\Loss(\param)
			- \frac{\Loss(\param)}2
			- |\Loss(\param)|\sqrt{
				\frac{1}{2^2}
				+ \frac{\|\nabla\Loss(\param)\|^2}{2a\Loss(\param)^2}
			}
		\right]\\
		\overset{\Loss(\param)<0}&{=}
		\underbrace{\Loss(\param)}_{<0}
		\underbrace{e^{-a\hat{\lr}^2}}_{\ge 1- a\hat{\lr}^2}
		\underbrace{\left[
			\frac12 + \sqrt{\frac14
			+ \frac{\|\nabla\Loss(\param)\|^2}{2a\Loss(\param)^2}}
		\right]}_{>0}\\
		&\le \Loss(\param)(1-a\hat{\lr}^2)
		\frac12\left[1 + \sqrt{1 + \frac{2}{aQ^2}}\right],
	\end{align*}
	where we use \(Q:= \frac{\Loss(\param)}{\|\nabla\Loss(\param)\|}<0\) again.
	Notice that
	\begin{align*}
		\hat{\lr}
		= |Q|\left(-\frac12 + \sqrt{\frac14 + \frac{1}{2aQ^2}}\right)
		= \frac{|Q|}2\left(-1 + \sqrt{1 + \frac{2}{aQ^2}}\right).
	\end{align*}
	We therefore want to use the third binomial formula by splitting \(\hat{\lr}^2\)
	\begin{align*}
		&\BLUE[\Loss(\param + \hat{\step})\mid \Loss(\param), \nabla\Loss(\param)]\\
		&\le \frac{\Loss(\param)}2\left(
			\left[1 + \sqrt{1+ \frac{2}{aq^2}}\right]
			- \frac{aQ^2}4\left(-1 + \sqrt{1 + \frac{2}{aQ^2}}\right)^2
			\left[1 + \sqrt{1 + \frac{2}{aQ^2}}\right]
		\right)\\
		&\le \frac{\Loss(\param)}2\left(
			\left[1 + \sqrt{1 + \frac{2}{aQ^2}}\right]
			- \frac{\bcancel{aQ^2}}4\left(-1 + \sqrt{1 + \frac{2}{aQ^2}}\right)
			\left[\left|\cancel{1}+ \frac{2}{\bcancel{aQ^2}}\right| - \cancel{1^2}\right]
		\right)\\
		&\le \frac{\Loss(\param)}2\left[
			\frac{3}2 + \frac12\sqrt{1+\frac2{aQ^2}}
		\right].
	\end{align*}
	Substituting \(Q\) with its definition and moving \(\frac12\) out of the
	bracket yields the claim.
\end{proof}

\subsection{Schlather-Moreva}

To bridge bounded variograms (stationary random fields) and
unbounded variograms (intrinsically stationary but not stationary random fields)
\textcite{schlatherParametricModelBridging2017} suggest
\begin{equation*}
	\variogram_{\alpha,\beta}(\step)
	= \frac{(1+\|\step\|^\alpha)^{\beta/\alpha}-1}{2^{\beta/\alpha}-1}
	\quad \alpha\in (0,2],\ \beta\in(-\infty, 2].
\end{equation*}
Since the model only results in a differentiable random field for \(\alpha=2\),
we will only consider \(\variogram_\beta(\step) := \variogram_{2,\beta}(\step) = \phi(\|\step\|^2)\),
with
\begin{equation*}
	\phi(x) = \frac{(1+x)^{\beta/2} -1}{2^{\beta/2}-1}.
\end{equation*}
Taking its derivative results in
\begin{equation*}
	\phi'(x) = \frac{\tfrac\beta2 (1+x)^{\tfrac\beta2 -1}}{2^{\beta/2}-1}
\end{equation*}
which implies that
\begin{align*}
	\hat{\lr} = \argmax_\lr \lr\frac{\phi'(\lr^2)}{\phi'(\lr)}
	= \argmax_\lr \lr(1+\lr^2)^{\tfrac\beta2 -1}.
\end{align*}
For \(\beta \ge 1\) this function is monotonically increasing in \(\lr\) and
we would therefore have \(\hat{\lr} = \infty\), so only the case \(\beta <1\)
is interesting\fxwarning{explanation for \(\beta\ge 1\)?}. Taking the derivative with regard to
\(\lr\) results in
\begin{align*}
	0 \overset!=\frac{d}{d\lr}
	&= (1+\lr^2)^{\frac\beta2 -2} \left[(1+\lr^2) - 2\lr^2\Big(1-\frac\beta2\Big)\right]\\
	&= (1+\lr^2)^{\frac\beta2 -2} \left[1 - \lr^2(1-\beta)\right]
\end{align*}
So we get
\[
	\hat{\lr} = \frac1{\sqrt{1-\beta}}.
\]
As we can see that the derivative is larger than zero before and smaller
afterwards it is indeed a maximum.
