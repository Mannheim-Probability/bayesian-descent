\section{Removing the `Centered' Assumption}

Another way to motivate \(\Loss\) being a random field is: Assume we have some
random field \(Z\), and randomly pick examples
\((X,Y) := (X,Z(X))\) from it. The real loss is the conditional expectation on
\(Z\)
\begin{equation*}
	\Loss(\param) = \E[\loss(\param, X,Y)\mid Z].
\end{equation*}
Then \(\Loss\) is a random variable for every \(\param\), i.e. a random field.
As an example, for the mean squared loss using some parametrized model
\(\model{X}\), we get
\begin{equation*}
	\Loss(\param)
	= \E[\|\model{X} - Y\|^2 \mid Z]
	= \E[\|\model{X} - Z(X)\|^2 \mid Z].
\end{equation*}
To actually work with this model, we need to make some assumptions.

\begin{axiom}
	\(Z\) is a centred stationary random field, \(X\) and \(Z\) are independent.
\end{axiom}

If we define
\begin{equation*}
	\mu(\param, X):= \E[\loss(\param, X,Y)\mid X],
\end{equation*}
e.g. for the mean squared case
\begin{align*}
	\mu(\param, X)&= \E[\|\model{X}-Z(X)\|^2 \mid X]\\
	&= \|\model{X}\|^2 - 2\langle \model{X}, \underbrace{\E[Z(X)\mid X]}_{=0}\rangle
	+ \underbrace{\E[\|Z(X)\|^2\mid X]}_{=\sigma^2}\\
	&= \|\model{X}\|^2 + \sigma^2,
\end{align*}
then we have by the tower property of conditional expectation
\begin{equation*}
	\E[\Loss(\param)]
	= \E[\loss(\param, X,Y)]
	= \E[\mu(\param, X)].
\end{equation*}
This allows us to center the loss. Or in the mean squared case: if we use no
information about \(Z\), we can still make sure that \(\Loss\) has constant
expectation \(\sigma^2\), by subtracting \(\model{X}^2\).