\section{Removing the `Centered' Assumption}

\subsection{Constant Mean}\label{subsec: constant mean}\fxnote{Should this be explained earlier?}

A cheap generalization is \(\E[\Loss(\param)]=\mu\) for a constant \(\mu\).
If we consider \(\bar{\Loss}(\param):=\Loss(\param) - \mu\) all the results
hold for \(\bar{\Loss}\) so
\begin{equation*}
	\BLUE[\Loss(\param+\step) \mid \Loss(\param), \grad\Loss(\param)]
	= \mu + \BLUE[\bar{\Loss}(\param+\step) \mid \bar{\Loss}(\param), \grad\Loss(\param)].
\end{equation*}
Minimization of the sum implies minimization of the second term, so we can recycle all results
but replace \(\bar{\Loss}\) with \(\Loss - \mu\) at the end. I.e. instead of 
\(\frac{\bar{\Loss}(\param)}{\|\nabla\Loss(\param)\|}\) we want to use
\(\frac{\Loss(\param) -\mu}{\|\nabla\Loss(\param)\|}\) in the learning rate
of Theorem~\ref{thm: bayesian descent}.

\subsection{Non-constant Mean}

A way to motivate \(\Loss\) being a random field is: Assume we have some
random field \(Z\), and randomly pick examples
\((X,Y) := (X,Z(X))\) from it. The real loss is the conditional expectation on
\(Z\)
\begin{equation*}
	\Loss(\param) = \E[\loss(\param, X,Y)\mid Z].
\end{equation*}
Then \(\Loss\) is a random variable for every \(\param\), i.e. a random field.
As an example, for the mean squared loss using some parametrized model
\(\model{X}\), we get
\begin{equation*}
	\Loss(\param)
	= \E[\|\model{X} - Y\|^2 \mid Z]
	= \E[\|\model{X} - Z(X)\|^2 \mid Z].
\end{equation*}
To actually work with this model, we need to make some assumptions.

\begin{axiom}
	\(Z\) is a centred stationary random field, \(X\) and \(Z\) are independent.
\end{axiom}

If we define
\begin{equation*}
	m_\Loss(\param, X):= \E[\loss(\param, X,Y)\mid X],
\end{equation*}
e.g. for the mean squared case
\begin{align*}
	m_\Loss(\param, X)&= \E[\|\model{X}-Z(X)\|^2 \mid X]\\
	&= \|\model{X}\|^2 - 2\langle \model{X}, \underbrace{\E[Z(X)\mid X]}_{=0}\rangle
	+ \underbrace{\E[\|Z(X)\|^2\mid X]}_{=\sigma_Z^2}\\
	&= \|\model{X}\|^2 + \sigma_Z^2.
\end{align*}
Where we have used that \(\E[Z(x)]=0\) and the independence of \(Z\) and \(X\)
for \(\E[Z(X)\mid X]=0\) and similar for the variance of \(Z\).

By the tower property of conditional expectation
\begin{equation*}
	m_\Loss(\param) := \E[\Loss(\param)]
	= \E[\loss(\param, X,Y)]
	= \E[m_\Loss(\param, X)]
	= \E\|\model{X}\|^2 + \sigma_Z^2.
\end{equation*}
While the constant \(\sigma_Z^2\) is no big deal and can be estimated from the
data \(Y=Z(X)\), the term \(\E\|\model{X}\|^2\) is more complicated. We have
\begin{equation*}
	\BLUE[\Loss(\param+\step)\mid \Loss(\param),\grad\Loss(\param)]
	= m_\Loss(\param+\step)
	+ \BLUE[\bar{\Loss}(\param+\step)\mid \bar{\Loss}(\param), \grad\bar{\Loss}(\param)]
\end{equation*}
Now even if we could calculate \(m_\Loss(\param)\) easily, to get \(\bar{\Loss}\)
and \(\grad\bar{\Loss}\), we then still have to minimize the sum. But the first
part \(m_\Loss\) is a deterministic function. And we started out trying to minimize
a deterministic function in our motivation and realized that we got nowhere. So
we can not really hope that we will get any better results for a black-box here.

\subsubsection{The Linear Model}

Assume that our input vector \(X\) has centered iid entries \(X^{(1)},\dots,X^{(\dimension)}\)
with variance \(\sigma_X^2\) and consider the linear model
\begin{equation*}
	\model[(\param, \param_0)]{X} = X\param + \tfrac{\sigma_X}{\sqrt{\dimension}}\param_0,
\end{equation*}
where we have split our usual parameters \(\param\) into coefficients \(\param\) and
bias \(\param_0\). Do not worry about the factor in the bias for now. Then we have
\begin{align*}
	\E\|\model[(\param, \param_0)]{X}\|^2
	&= \param^T \E[X^T X] \param
	+ 2\param_0\tfrac{\sigma_X}{\sqrt{\dimension}} \langle 1, \underbrace{\E[X]}_{=0}\param\rangle
	+ \param_0^2\underbrace{\tfrac{\sigma_X^2}{\dimension}\langle 1, 1\rangle }_{=\sigma_X^2}\\
	&=\sigma_X^2 (\|\param\|^2 + \param_0^2)
\end{align*}
Joining \(\param\) and \(\param_0\) back together we have
\[
	m_\Loss(\param) = \sigma_X^2 \|\param\|^2 + \sigma_Z^2.
\]
Now you probably sopped wondering about the weird bias term.

\subsubsection{\texorpdfstring{\(L_2\)}{L₂}-Regularization}

Now that we motivated \(m_\Loss(\param) = \sigma_X^2\|\param\|^2 + \sigma_Z^2\), we
will forget its origin and simply take it as a given. The result we will derive
from this, will essentially be some form of \(L_2\)-regularization of our
optimization problem.


