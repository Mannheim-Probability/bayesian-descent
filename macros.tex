%% AMS Theorem Environments

%% REPEATABLE THEOREMS
%% cf. https://tex.stackexchange.com/a/443/148164
\makeatletter
\newcommand{\newreptheorem}[2]{% define repeatable theorems
	\newtheorem*{rep@#1}{\rep@title} %  -> no number, special title

	\newenvironment{rep#1}[1]{% define repetition environement with required label-reference
		\def\rep@title{#2 \ref{##1}} % use reference to define title
		\begin{rep@#1} % begin second occasion with correct title definition
	}{
		\end{rep@#1}
	}
}
\makeatother

\newreptheorem{test}{Test}
\theoremstyle{plain}% default
\newtheorem{prop}{Proposition}[section]
\newtheorem{lemma}[prop]{Lemma}
\newreptheorem{lemma}{Lemma}
\newtheorem{corollary}[prop]{Corollary}
\newtheorem{theorem}[prop]{Theorem}

\theoremstyle{definition}
\newtheorem{definition}[prop]{Definition}
\newtheorem{example}[prop]{Example}
\newtheorem{assumption}[prop]{Assumption}
\newtheorem{axiom}[prop]{Axiom}

\theoremstyle{remark}
\newtheorem{remark}[prop]{Remark}



%%
\newcommand*{\grad}{\nabla}
\newcommand*{\hessian}{\nabla^2}

\newcommand*{\real}{\mathbb{R}}
\newcommand*{\dimension}{d}

\newcommand*{\Rom}[1]{\text{\RN{#1}}} % uppercase roman numbers
\newcommand*{\rom}[1]{\text{\Rn{#1}}} % lowercase roman numbers

\newcommand*{\E}{\mathbb{E}}
\renewcommand{\Pr}{\mathbb{P}}
\DeclareMathOperator{\sgn}{sgn}
\newcommand*{\identity}{\mathbb{I}}
\DeclareMathOperator{\diag}{diag}
\newcommand*{\Loss}{\mathcal{L}}
\newcommand*{\loss}{\ell}
\newcommand*{\param}{\theta}
\newcommand{\model}[2][\param]{f(#1, #2)}
\DeclareMathOperator*{\argmin}{argmin}
\DeclareMathOperator*{\argmax}{argmax}

\newcommand*{\step}{\mathbf{d}}
\newcommand*{\lr}{\eta}
\newcommand*{\ubound}{L}
\DeclareMathOperator{\LUE}{\hyperref[def: LUE]{LUE}} % Set of Linear Unbiased Estimators
\DeclareMathOperator{\BLUE}{\hyperref[def: BLUE]{BLUE}} % Best Linear Unbiased Estimator
\DeclareMathOperator{\EI}{EI} % Expected Improvement
\DeclareMathOperator{\KG}{KG} % Knowledge Gradient
\newcommand*{\variogram}{\hyperref[def: variogram]{\gamma}}


\DeclareMathOperator{\linHull}{Lin}
\DeclareMathOperator{\Cov}{Cov}
\DeclareMathOperator{\Var}{Var}
\newcommand*{\C}{\hyperref[def: covariance function]{\mathcal{C}}} % Covariance function
\newcommand*{\sqC}{C} % isotropic Covariance function
